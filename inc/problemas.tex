\documentclass[12pt]{article}
\usepackage[utf8]{inputenc}
\usepackage[spanish]{babel}
\usepackage{titlesec}
\usepackage[official]{eurosym}
\usepackage{siunitx} \sisetup{ output-decimal-marker={,}, quotient-mode=fraction}   % Sistema internacional para unidades con separador decimal
\usepackage[hidelinks]{hyperref}                    % Enlaces sin apariencia fea
\usepackage{enumitem}

\title{Recopilatorio de problemas de matemáticas}
\author{Samuel Gómez}
\date{\today}


% Entorno problema ========================================
\newcounter{problema}[section]
\newenvironment{problema}[3]
{\refstepcounter{problema} \noindent\textbf{Problema \theproblema} \hspace{0.5cm} \textbf{#1}
	\paragraph{} #2

	\ifdefined\soluciones {
		\paragraph{Solución} #3 ..
	} \else {} \fi
}
{\vspace{1cm}}

% Cambio sección ==========================================
\titleformat{\section}
  {\normalfont\Large\bfseries}   % The style of the section title
  {}                             % a prefix
  {0pt}                          % How much space exists between the prefix and the title
  {}    						 % How the section is represented


% = BEGIN DOCUMENT ======================================================
\begin{document}
% soluciones.cfg puede contener: \def\soluciones{true}
% en ese caso se muestran las soluciones de los problemas
% en caso de que el fichero esté vacío -> no se define la macro y no se imprimen las soluciones
%\def\soluciones{true}


\maketitle
\vspace{2cm}
\tableofcontents
\newpage





% == NIVEL ============================================
\section{Nivel I}
\begin{problema}
{El despistado}
{Marcos ha perdido el 35\% del dinero que llevaba. Si ahora tiene 240\euro{}, ¿cuánto dinero
llevaba la principio y cuanto ha perdido?}
{}
\end{problema}

\begin{problema}
{Ruta en ascenso}
{En una ruta ascendemos \SI{100}{\meter} por cada $\SI{3}{\kilo\meter}$ recorridos.
	\begin{enumerate}[label=\alph*)]
	\item ¿Cuál es la razón entre los metros ascendidos y los recorridos?
	\item Si hemos recorrido $\SI{12}{\kilo\meter}$, ¿cuántos metros hemos ascendido?
	\end{enumerate}
}
{}
\end{problema}

\begin{problema}
	{El lado del cuadrado}
	{Si el lado de una cuadrado es el doble que el de otro, ¿es también el área del primer cuadrado
	doble que el segundo? Pon un ejemplo y razona tu respuesta.}
{}
\end{problema}

\begin{problema}
	{El vendedor de coches}
	{Pedro cobra mensualmente 1200 \euro{} de sueldo fijo más 300 \euro{} por cada coche que vende.
	
	\begin{enumerate}[label=\alph*)]
		\item Escribe la fórmula de la función que representa el sueldo mensual en función del
			número de coches que vende. ¿De qué tipo es?
		\item Representa gráficamente la función
		\item Si un mes vende 3 coches, ¿cuál será su sueldo?
		\item Si ha cobrado 3000 \euro{} este mes, ¿cuántos coches ha vendido?
		\item Indica si esta función es creciente o decreciente y si tiene algún máximo o mínimo
			relativo
	\end{enumerate}
	}
	{}
\end{problema}

\begin{problema}
	{Las hermanas}
	{Patricia dice: "Mi hermana tenía 4 años cuando yo nací."
	\begin{enumerate}[label=\alph*)]
		\item Expresa con una función como varía la edad de Patricia al variar la de su hermana. Haz
			una tabla de valores de la función obtenida y represéntala gráficamente.
		\item ¿Cuántos años tendrá Patricia cuando su hermana le doble la edad?
	\end{enumerate}
	}
	{}
\end{problema}


\begin{problema}
	{La ahorradora}
	{Tania decide ahorrar 3 \euro{} semanales. Encuentra la fórmula que relaciona el dinero ahorrado
	con el número de semanas.
	\begin{enumerate}[label=\alph*)]
		\item ¿Cuánto habrá ahorrado en 11 semanas?
		\item Calcula el número de semanas que tienen que transcurrir para que tenga ahorrados
			78\euro{}
	\end{enumerate}
	}
	{}
\end{problema}


% == NIVEL ============================================
\newpage
\section{Nivel II}
\begin{problema}
{Ramo de flores}
{Un ramo de 9 flores iguales cuesta 13,5\euro{}. Halla el precio de un ramo de las mismas flores
que tenga 4 más que el anterior.}
{}
\end{problema}


\begin{problema}
{Coser pantalones}
{Tres personas cosen 12 pantalones en 10 minutos. Halla cuántos pantalones coserán 4 personas en 15
	minutos.}
{}
\end{problema}

\begin{problema}
{Nacimientos}
{En un titular de un periódico se afirmaba: "los nacimientos caen casi un 30\% en España en la
	última década". En 2008, el número de nacimientos en España fue de 519.799 y en 2018, de
	369.302. Comprueba si es cierta la afirmación del titular, calculando el porcentaje exacto.
}
{}
\end{problema}

% == NIVEL ============================================
\newpage
\section{Nivel III}

\begin{problema}
{Miguel y su primo}
{Miguel tiene ahora cuatro años más que su primo Ignacio y, dentro de tres años, entre
los dos sumarán 20 años. ¿Cuántos años tiene cada uno?}
{}
\end{problema}

\begin{problema}
{Un problema capital}
{un padre cede a un hijo un quinto de su capital, a otro un cuarto y a un tercer hijo le da
el resto, que son 19.800 \euro{} ¿cuál era su capital?.}
{}
\end{problema}


\begin{problema}
{Dos números consecutivos}
{Halla dos números consecutivos, tales que añadiendo al cuadrado del mayor la mitad del menor resulta
27.}
{}
\end{problema}





% == NIVEL ============================================
\newpage
\section{Nivel IV}
\begin{problema}{Juguetes}
{En una tienda hay dos tipos de juguetes, los de tipo A que utilizan 2 pilas y los de tipo B que utilizan 5 pilas. Si en total en
la tienda hay 30 juguetes y 120 pilas, ¿cuántos juguetes hay de cada tipo?}
{}
\end{problema}


\begin{problema}{Las notas}
{Nieves le pregunta a Miriam por sus calificaciones en Matemáticas y en Lengua. Miriam le dice "La suma de mis
calificaciones es 19 y el producto 90". Nieves le da la enhorabuena. ¿Qué calificaciones obtuvo?}
{}
\end{problema}


\begin{problema}{Master chef}
{Se desea mezclar harina de 2 \euro{}/kg con harina de 1 \euro{}/kg para obtener una mezcla de 1.2
\euro{}/kg. ¿Cuántos kg deberemos
poner de cada precio para obtener 300 kg de mezcla?}
{}
\end{problema}


\begin{problema}
{Virus y ballenas}
	{El ser vivo más pequeño es un virus que pesa del orden de $\SI{10d-10}{\gram}$ y el más grande es la ballena azul, que pesa,
	aproximadamente, 138 toneladas. ¿Cuántos virus serían necesarios para conseguir el peso de la ballena?.}
{}
\end{problema}

% == NIVEL ============================================
\newpage
\section{Nivel V}
\begin{problema}
{Mitades y caudrados}
{La mitad de un número entero sumado con su cuadrado es igual a ocho veces dicho número menos 9.
	Calcula dicho número.}
{}
\end{problema}

\begin{problema}
{Excesos de catetos}
{En un triángulo rectángulo, un cateto excede en 7 unidades al otro. Si sabemos que su hipotenusa es
	85, ¿cuánto mide cada cateto?}
{}
\end{problema}

\begin{problema}
{La raíz cuadrada que se creía inferior}
{Calcula un número sabiendo que la suma de dicho número con el doble de su raíz cuadrada es inferior
	en 3 unidades al doble de dicho número.}
{}
\end{problema}

\begin{problema}
{Superar el examen de matemáticas}
{Para superar la asignatura de matemáticas hay tres pruebas distintas: un examen teórico, uno de
	problemas y una exposición de un trabajo. Pedro, Juana y Pablo se han examinado y han obtenido
	las notas siguientes:
	\begin{center}
	\begin{tabular}{ | l | c  c  c | c |} 
	 \hline
		& \textbf{Teoría} & \textbf{Problemas} & \textbf{Exposición} & \textbf{Nota final} \\
	 \hline
		\textbf{Pedro} & 8 & 6 & 4 & 6,2 \\
		\textbf{Juana} & 7 & 5 & 8 & 5,95 \\
		\textbf{Pablo} & 3 & 9 & 7 & 7,2 \\
	 \hline
	\end{tabular}
	\end{center}
	¿Qué porcentaje de la nota final corresponde a cada una de las pruebas?
}
{}
\end{problema}

\newpage
\begin{problema}
{Padre e hijo}
{La edad de un padre excede en 3 unidades el doble de la edad de su hijo.
Calcula la edad del padre y del hijo sabiendo que dentro de tres años el doble de la edad
	del padre superará en 24 años al triple de la edad del hijo.}
{}
\end{problema}


\begin{problema}
{Rectángulo y cuadrado}
{El área de un cuadrado es la sexta parte del área de un rectángulo.
Además, el lado del cuadrado es la mitad de uno de los lados del rectángulo y el perímetro
del rectángulo es de $\SI{30}{\centi\meter}$. Calcula las dimensiones del cuadrado y del
	rectángulo.}
{}
\end{problema}
%\begin{problema}{Título}{Cuerpo}\end{problema}



\end{document}
