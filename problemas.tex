\documentclass[a4paper]{book}
%\documentclass[a4paper,twocolumn]{book}
%\setlength{\columnsep}{0.5cm}
\usepackage[utf8]{inputenc}
\usepackage[spanish]{babel}
\usepackage{titlesec}
\usepackage[official]{eurosym}
\usepackage{siunitx} \sisetup{ output-decimal-marker={,}, quotient-mode=fraction}   % Sistema internacional para unidades con separador decimal
\usepackage[hidelinks]{hyperref}                    % Enlaces sin apariencia fea
\usepackage{enumitem}
\usepackage{geometry}
%\geometry{a4paper, textwidth=17cm}

\title{Recopilatorio de problemas de matemáticas}
\author{Samuel Gómez}
\date{\today}


% Entorno problema ========================================
%\begin{problema}{Enunciado}\end{problema}
\newcounter{problema}[section]
\newenvironment{problema}[2]
{\refstepcounter{problema} \noindent\textbf{Problema \thesection.\theproblema} \hspace{0.5cm} #1
	\ifdefined\soluciones {
		\paragraph{Solución} #2 ..
	} \else {} \fi
}
{\vspace{1cm}}

% Cambio sección ==========================================
\titleformat{\section}
  {\normalfont\Large\bfseries}   % The style of the section title
  {\thesection.- }                             % a prefix
  {0pt}                          % How much space exists between the prefix and the title
  {}    						 % How the section is represented


% = BEGIN DOCUMENT ======================================================
\begin{document}
% soluciones.cfg puede contener: \def\soluciones{true}
% en ese caso se muestran las soluciones de los problemas
% en caso de que el fichero esté vacío -> no se define la macro y no se imprimen las soluciones
%\input{soluciones.cfg}



\pagenumbering{gobble}
\maketitle

\pagenumbering{Roman}
\tableofcontents

\pagenumbering{arabic}
\chapter{Problemas de 1º ESO}

\section{Aritmética}

\begin{problema}
{
Marcos ha perdido el 35\% del dinero que llevaba. Si ahora tiene 240 \euro{}, ¿cuánto dinero
llevaba la principio y cuanto ha perdido?
}{}\end{problema}

\begin{problema}
{
En una ruta ascendemos 100 metros por cada 3 kilómetros recorridos.
\begin{enumerate}[label=\alph*)]
  \item  ¿Cuál es la razón entre los metros ascendidos y los recorridos?
  \item  Si hemos recorrido 12 kilómetros, ¿cuántos metros hemos ascendido?
\end{enumerate}
}{}\end{problema}

\begin{problema}
{
Si el lado de una cuadrado es el doble que el de otro, ¿es también el área del primer cuadrado
doble que el segundo? Pon un ejemplo y razona tu respuesta.
}{}\end{problema}

\begin{problema}
{
Pedro cobra mensualmente 1200\euro{} de sueldo fijo más 300\euro{} por cada coche que vende.
\begin{enumerate}[label=\alph*)]
  \item  Escribe la fórmula de la función que representa el sueldo mensual en función del 
  número de coches que vende. ¿De qué tipo es?
  \item  Representa gráficamente la función
  \item  Si un mes vende 3 coches, ¿cuál será su sueldo?
  \item  Si ha cobrado 3000\euro{} este mes, ¿cuántos coches ha vendido?
  \item  Indica si esta función es creciente o decreciente y si tiene algún máximo o mínimo relativo
\end{enumerate}
}{}\end{problema}

\begin{problema}
{
Patricia dice: «Mi hermana tenía 4 años cuando yo nací»
\begin{enumerate}[label=\alph*)]
  \item  Expresa con una función como varía la edad de Patricia al variar la de su hermana. Haz
      una tabla de valores de la función obtenida y represéntala gráficamente.
  \item  ¿Cuántos años tendrá Patricia cuando su hermana le doble la edad?
\end{enumerate}
}{}\end{problema}

\begin{problema}
{
Tania decide ahorrar 3\euro{} semanales. Encuentra la fórmula que relaciona el dinero ahorrado 
con el número de semanas.
\begin{enumerate}[label=\alph*)]
  \item  ¿Cuánto habrá ahorrado en 11 semanas?
  \item  Calcula el número de semanas que tienen que transcurrir para que tenga ahorrados 78\euro{}
\end{enumerate}

}{}\end{problema}

\chapter{Problemas de 2º ESO}

\section{Aritmética}

\begin{problema}
{
Un ramo de 9 flores iguales cuesta 13,5\euro{}. Halla el precio de un ramo de las mismas flores
que tenga 4 más que el anterior.
}{}\end{problema}

\begin{problema}
{
Tres personas cosen 12 pantalones en 10 minutos. Halla cuántos pantalones coserán 4 personas en 15
minutos.
}{}\end{problema}

\begin{problema}
{
En un titular de un periódico se afirmaba: «los nacimientos caen casi un 30\% en España en la
última década». En 2008, el número de nacimientos en España fue de 519.799 y en 2018, de
369.302. Comprueba si es cierta la afirmación del titular, calculando el porcentaje exacto.
}{}\end{problema}

\begin{problema}
{
En un cruce hay tres semáforos que cambian de color cada 54 segundos, 60 segundos y 72 segundos. Si
a las 09:00 coinciden en un cambio, ¿cuánto tiempo ha de transcurrir para que vuelvan a coincidir?
}{}\end{problema}

\begin{problema}
{
Queremos hacer lotes iguales con tres artículos, $A$, $B$ y $C$. Tenemos 276 unidades de $A$, 300 de
$B$ y 312 de $C$. ¿Cuál es el mayor número de lotes que podemos hacer sin que sobre ninguna
unidad?¿Cuál es su composición?
}{}\end{problema}

\begin{problema}
{
Sales de casa con 50\euro{}. Compras tres entradas oara ek cine de 8\euro{} cada una y un libro de 19\euro{}. Tus
amigos te pagan sus entradas y gastas 3\euro{} en un refresco. Expresa el dinero que te queda con una
operación combinada.
}{}\end{problema}

\begin{problema}
{
Margarita sale de casa con cierta cantidad de dinero. Encuentra a su amigo Carlos y le paga 28\euro{} que
le debía. Después, se enceuntra a su tía, que le regala 60\euro{} por su cumpleaños. Se gasta 43\euro{} en cosas
que necesita y vuelve a casa con 26\euro{}. ¿Con qué dinero había salido de casa?
}{}\end{problema}

\begin{problema}
{
Ana compra veintiocho juegos de vídeoconsola por 896\euro{} y los vende a 56\euro{} cada uno.
\begin{enumerate}[label=\alph*)]
  \item  ¿Cuántos juegos tien que vender para ganar 168\euro{}?
  \item  ¿Cuánto gana si los vende todos?
  \item  ¿Cuántos ha vendido si ha ganado 280\euro{}?
\end{enumerate}
}{}\end{problema}

\begin{problema}
{
Un carnicero compra una cámara frigorífica que es capaz de reducir la temperatura en 6 ºC cada hora.
Al conectarla, marca una temperatura inicial de 2ºC.
\begin{enumerate}[label=\alph*)]
  \item  ¿Qué temperatura habrá en el interior de la cámara frigorífica después de 5 h?
  \item  Si la temperatura mínima de la cámara frigorífica es de $-42$ºC, ¿cuánto tiempo tardará en
\end{enumerate}
alcanzarla?
}{}\end{problema}

\begin{problema}
{
Una empresa tiene 3.000 \euro{} de fondo al comenzar sus actividades. Si los ingresos diarios son de 150 \euro{}
y los gastos diarios son de 165 \euro{}, ¿en cuántos días se agotará el fondo?
}{}\end{problema}

\begin{problema}
{
Un librero de lance compra 145 libros saldados a 15 \euro{} cada libro. Además, la editorial le regala
otros dos libros. ¿A cuánto debe vender cada uno de los libros si desea obtener 1.500 \euro{} de
beneficio?
}{}\end{problema}

\begin{problema}
{
En una familiar, el padre gana 63 \euro{} diarios y la madre, 75 \euro{} diarios. Si los gastos familiares son
de 103 \euro{} diarios, ¿qué expresión permite calcular cuánto ahorran en 30 días de trabajo? Razona la
respuesta.
\begin{enumerate}[label=\alph*)]
  \item  $30\cdot(103-(75+63))$
  \item  $30\cdot 103 - (75+63)$
  \item  $30\cdot (75+63-103)$
  \item  $30\cdot 75 + 30\cdot 63-103$
\end{enumerate}
}{}\end{problema}

\begin{problema}
{
Disponemos deuna caja de 90 cm de largo, 54 cm de ancho y 72 cm de alto. Queremos rellenarla con
cubos de madera del mayor tamaño posible sin que sobre espacio.
\begin{enumerate}[label=\alph*)]
  \item  ¿Qué tamaño deben tener estos cubos?
  \item  ¿Cuántos cubos caben?
\end{enumerate}
}{}\end{problema}

\begin{problema}
{
Queremos alicatar una pared de 425 cm de largo por 250 cm de alto con azulejos cuadrados del mayor
tamaño posible, sin tener que cortar ninguno. ¿Cuántos azulejos necesitamos?
}{}\end{problema}

\begin{problema}
{
Un carpintero tiene tres listones, de 144 cm, 80 cm y 192 cm de longitud. Desea dividirlos en trozos
de igual longitud y que sea la mayor posible. ¿De qué logitud será cada trozo? ¿Cuántos trozos podrá
hacer?
}{}\end{problema}

\begin{problema}
{
¿Cuál es el mayor número de lotes que se pueden hacer con 345 chocolatinas, 375 bolsas de caramelos
y 390 sobres de frutos secos de forma que cada lote tenga la misma composición y no sobre nada?

}{}\end{problema}

\chapter{Problemas de 3º ESO}

\section{Álgebra}

\begin{problema}
{
Miguel tiene ahora cuatro años más que su primo Ignacio y, dentro de tres años, entre
los dos sumarán 20 años. ¿Cuántos años tiene cada uno?
}{}\end{problema}

\begin{problema}
{
un padre cede a un hijo un quinto de su capital, a otro un cuarto y a un tercer hijo le da
el resto, que son 19.800\euro{} ¿cuál era su capital?
}{}\end{problema}

\begin{problema}
{
Halla dos números consecutivos, tales que añadiendo al cuadrado del mayor la mitad del menor 
resulta 27.
}{}\end{problema}

\chapter{Problemas de 4º ESO}

\section{Álgebra}

\begin{problema}
{
En una tienda hay dos tipos de juguetes, los de tipo $A$ que utilizan 2 pilas y los de tipo $B$ que utilizan 
5 pilas. Si en total en la tienda hay 30 juguetes y 120 pilas, ¿cuántos juguetes hay de cada tipo?
}{}\end{problema}

\begin{problema}
{
Nieves le pregunta a Miriam por sus calificaciones en Matemáticas y en Lengua. Miriam le dice «La suma de mis
calificaciones es 19 y el producto 90». Nieves le da la enhorabuena. ¿Qué calificaciones obtuvo?
}{}\end{problema}

\begin{problema}
{
Se desea mezclar harina de 2 \euro{}/kg con harina de 1 \euro{}/kg para obtener una mezcla de 1,2 \euro{}/kg
¿Cuántos kg deberemos poner de cada precio para obtener 300 kg de mezcla?
}{}\end{problema}

\begin{problema}
{
El ser vivo más pequeño es un virus que pesa del orden de $\SI{10d-10}{\gram}$ y el más grande es la ballena azul, que pesa,
aproximadamente, 138 toneladas. ¿Cuántos virus serían necesarios para conseguir el peso de la ballena?.

}{}\end{problema}

\chapter{Problemas de 1º Bachillerato}

\section{Álgebra}

\begin{problema}
{
Se quiere vallar un campo rectangular. Se sabe que uno de sus lados mide tres quintas partes de la
medida del otro y la diagonal mide 30 metros. Si un metro de valla cuesta 25\euro{} y se desperdicia
un 10\% del material empleado, calcula el precio que se deberá pagar.
}{}\end{problema}

\begin{problema}
{
Tres números reales positivos $a$, $b$ y $c$ pueden representar las medidas de los tres lados de un
triángulo si y solo so verifican que la suma de los dos menores es mayor que el menor. Indica los
valores que pueden tomar $a$ para que los números naturales $a$, $2a+1$ y 10 sean las medidas de un
triángulo.
}{}\end{problema}

\begin{problema}
{
La mitad de un número entero sumado con su cuadrado es igual a ocho veces dicho número menos 9.
Calcula dicho número.
}{}\end{problema}

\begin{problema}
{
En un triángulo rectángulo, un cateto excede en 7 unidades al otro. Si sabemos que su hipotenusa es
85, ¿cuánto mide cada cateto?
}{}\end{problema}

\begin{problema}
{
Calcula un número sabiendo que la suma de dicho número con el doble de su raíz cuadrada es inferior
en 3 unidades al doble de dicho número.
}{}\end{problema}

\begin{problema}
{
La edad de un padre excede en 3 unidades el doble de la edad de su hijo. Calcula la edad del padre y 
del hijo sabiendo que dentro de tres años el doble de la edad del padre superará en 24 años al triple 
de la edad del hijo.
}{}\end{problema}

\begin{problema}
{
El área de un cuadrado es la sexta parte del área de un rectángulo. Además, el lado del cuadrado es 
la mitad de uno de los lados del rectángulo y el perímetro del rectángulo es de 30 cm. 
Calcula las dimensiones del cuadrado y del rectángulo.
}{}\end{problema}

\begin{problema}
{
Para superar la asignatura de matemáticas hay tres pruebas distintas: un examen teórico, uno de
problemas y una exposición de un trabajo. Pedro, Juana y Pablo se han examinado y han obtenido
las notas siguientes:
\begin{center}
\begin{tabular}{  l  c  c  c  c }
\hline
          & \textbf{Teoría} & \textbf{Problemas} & \textbf{Exposición} & \textbf{Nota final}\\
\hline
\textbf{Pedro} &      8     &        6      &       4        &       6,2     \\
\textbf{Juana} &      7     &        5      &       8        &       5,95    \\
\textbf{Pablo} &      3     &        9      &       9        &       7,2     \\
\hline
\end{tabular}
\end{center}
¿Qué porcentaje de la nota final corresponde a cada una de las pruebas?

}{}\end{problema}

\end{document}
